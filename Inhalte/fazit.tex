\section{Fazit}

VeriFast ist ein bekanntest Verifikationswerkzeug im Bereich der formalen Programmverifikation, dass bereits in unterschiedlichen Projekten eingesetzt und sogar an einigen Universitäten unterrichtet wird. Dementsprechend lassen sich zahlreiche Paper, Tutorials und Beispielprogramme in Bezug auf VeriFast finden. Gerade die enthaltene Bibliothek, welche viele hilfreiche Definitionen (Induktive Datentypen, Fixpunkt Funktionen etc.) liefert, kann an vielen Stellen während der Verifikation gewinnbringend eingesetzt werden. Die einprägsame Syntax der Sprachkonstrukte erlaubt eine schnelle Einarbeitung und einfache Definition neuer Konstrukte. Dies wird durch die VeriFast IDE ergänzt, welche einiges an Funktionalität zur Verfügung stellt und dadurch die Programmverifikation erheblich vereinfacht. Hier ist besonders die schrittweise Verifikation einzelner Ausführungspfade zu erwähnen, wodurch Programme genauer analysiert und mögliche Fehler in Definitionen schneller gefunden und behoben werden können. Auch die angezeigten Fehlermeldungen wurden sinnvoll gewählt und geben zuverlässige Hinweise auf mögliche Schwachstellen. Die Verifikation an sich läuft schnell und zuverlässig ab, dessen Ergebnis direkt dem Statusfeld entnommen werden kann.

Der Wunsch, bereits existierende C-Programme nachträglich zu verifizieren, ohne dabei Teile des Quellcodes anpassen zu müssen, ist in vielen Fällen nur schwer zu realisieren. Vielmehr sollte das Konzept der Programmverifikation frühst möglich in den Entwicklungsprozess einbezogen werden, um ggf. Sprachlimitationen von Verifikationswerkzeugen zu umgehen, sowie das Design-by-Contract Paradigma sauber und vollständig umzusetzen. Eine weiteres Problem ist das finden von Schleifeninvarianten, das gerade bei komplexen Programmen eine Herausforderungen darstellt. Zwar gibt es Werkzeuge die hierfür Hilfestellung anbieten, jedoch sind die Ergebnisse nur bedingt anwendbar \cite{Crocker2007}. Ansätze, dies zu Automatisieren, gibt es bereits in \cite{Leino2005} und \cite{Stark1990} und könnten in Tools wie VeriFast eingearbeitet werden. Auch gilt es zu beachten, das verifizierte Programme, die mit unterschiedlichen Compilern bzw. Compiler-Flags kompiliert wurde, oder auf unterschiedlicher Hardware laufen, unterschiedliche Ergebnisse liefern können, wodurch weitere Verifikationmechanismen notwendig werden. Ein Ansatz, in dem Model Checking für die Programmverifikation und Testfallgenerierung verfolgt wird, ist in \cite{Kandl2007} zu finden.